% Copyright 2004 by Till Tantau <tantau@users.sourceforge.net>.
%
% In principle, this file can be redistributed and/or modified under
% the terms of the GNU Public License, version 2.
%
% However, this file is supposed to be a template to be modified
% for your own needs. For this reason, if you use this file as a
% template and not specifically distribute it as part of a another
% package/program, I grant the extra permission to freely copy and
% modify this file as you see fit and even to delete this copyright
% notice. 

\documentclass{beamer}
\usepackage[absolute,overlay]{textpos}
\usepackage[timeinterval=1]{tdclock}


\usepackage{amssymb,amsmath,amsfonts,amsthm}
\usepackage{url}
\usepackage[backend=bibtex,style=verbose]{biblatex}
\footnotesize{\small}

\bibliography{s-arc-solCay.bib}

% There are many different themes available for Beamer. A comprehensive
% list with examples is given here:
% http://deic.uab.es/~iblanes/beamer_gallery/index_by_theme.html
% You can uncomment the themes below if you would like to use a different
% one:
%\usetheme{AnnArbor}
%\usetheme{Antibes}
%\usetheme{Bergen}
%\usetheme{Berkeley}
%\usetheme{Berlin}
%\usetheme{Boadilla}
%\usetheme{boxes}
%\usetheme{CambridgeUS}
%\usetheme{Copenhagen}
%\usetheme{Darmstadt}
%\usetheme{default}
%\usetheme{Frankfurt}
%usetheme{Goettingen}
%\usetheme{Hannover}
%\usetheme{Ilmenau}
%\usetheme{JuanLesPins}
%\usetheme{Luebeck}
\usetheme{Madrid}
%\usetheme{Malmoe}
%\usetheme{Marburg}
%\usetheme{Montpellier}
%\usetheme{PaloAlto}
%\usetheme{Pittsburgh}
%\usetheme{Rochester}
%\usetheme{Singapore}
%\usetheme{Szeged}
%\usetheme{Warsaw}


\usepackage{multirow}





\def\SL{\mathrm{SL}}
\def\GL{\mathrm{GL}}
\def\PSL{\mathrm{PSL}}
\def\PGL{\mathrm{PGL}}
\def\PSU{\mathrm{PSU}}
\def\PSp{\mathrm{PSp}}
\def\PO{\mathrm{P\Omega}}
\def\PGaL{\mathrm{P\Gamma L}}
\def\Aut{\mathrm{Aut}}
\def\Inn{\mathrm{Inn}}
\def\Out{\mathrm{Out}}
\def\Cay{\mathrm{Cay}}
\def\val{\mathrm{val}}
\def\HS{\mathrm{HS}}
\def\PH{\mathcal{PH}}
\def\GI{\mathcal{GI}}
\def\PG{\mathrm{PG}}




%%%%%%%%%%

%START

%%%%%%%%%%








\begin{document}


\title{s-Arc-transitive solvable Cayley graphs}
\setbeamerfont{subtitle}{size=\fontsize{8}{15}}
\subtitle{CAI HENG LI, JIANGMIN PAN, AND YINGNAN ZHANG}

% A subtitle is optional and this may be deleted
%\subtitle{PhD Thesis Defense}

\author{Speaker: Yuandong Li}
% - Give the names in the same order as the appear in the paper.
% - Use the \inst{?} command only if the authors have different
%   affiliation.

\institute[BJTU] % (optional, but mostly needed)
{Beijing Jiaotong University}

% - Use the \inst command only if there are several affiliations.
% - Keep it simple, no one is interested in your street address.

\date{\today}
% - Either use conference name or its abbreviation.
% - Not really informative to the audience, more for people (including
%   yourself) who are reading the slides online

%\subject{Applied Math}
% This is only inserted into the PDF information catalog. Can be left
% out. 

% If you have a file called "university-logo-filename.xxx", where xxx
% is a graphic format that can be processed by latex or pdflatex,
% resp., then you can add a logo as follows:

% \pgfdeclareimage[height=0.5cm]{university-logo}{university-logo-filename}
% \logo{\pgfuseimage{university-logo}}

% Delete this, if you do not want the table of contents to pop up at
% t`	he beginning of each subsection:
\AtBeginSubsection[]
{
  \begin{frame}<beamer>{Outline}
    \tableofcontents[currentsection,currentsubsection]
  \end{frame}
}




% Let's get started


\begin{frame}
\titlepage
%\initclock
%\date[\initclock \cronominutes\timeseparator\cronoseconds]{}
\end{frame}
\date{\crono}


\begin{frame}{Outline}
 \tableofcontents
\end{frame}

% Section and subsections will appear in the presentation overview
% and table of contents.


\section{Preliminaries \& Background}

\begin{frame}{Preliminaries}
Let $\Gamma$ be a finite simple undirected graph. Let $G\leq \Aut\Gamma$.
\begin{definition}
	\begin{itemize}
		\item \textbf{s-arc:} (s+1)-tuple of vertices $\alpha_0,\alpha_1,\cdots,\alpha_s$ where $\alpha_i$ is adjacent to $\alpha_{i+1}$ and $\alpha_{j-1}\neq\alpha_{j+1}$ for $0\leq i\leq s-1$ and $1\leq j\leq s-1$.
		\item \textbf{(G,s)-arc-transitive:} $G$ is transitive on the set of s-arcs of $\Gamma$.
		\item \textbf{(G,s)-transitive:} (G,s)-arc-transitive but not (G,s+1)-arc-transitive.
	\end{itemize}
For short, \textbf{s-transitive} means $(\Aut\Gamma,s)$-transitive for graphs.
\end{definition}
\begin{lemma}
	\begin{itemize}
		\item s-arc-transitive ($1\leq s$) $\Longrightarrow$ k-arc-transitive ($1\leq k\leq s$).
		\item the s-arc-transitivity of a graph is inherited by the normal quotients.
	\end{itemize} 
\end{lemma}
\end{frame}

\begin{frame}{Preliminaries}
\begin{definition}
	\begin{itemize}
		\item \textbf{Cayley graph:} $\Cay(G,S)$ with vertex set $G$ and edges $yx^{-1}\in S$.
		\item \textbf{solvable Cayley graph:} $G$ is solvable.
	\end{itemize}
\end{definition}
\begin{lemma}
	\center $\Gamma$ is Cayley of $G$ $\iff$ $\exists G\lesssim \Aut\Gamma$ which is vertex-regular.
\end{lemma}
\end{frame}


\begin{frame}{Background}
\begin{itemize}
	\item 1947, Tutte: No 6-arc-transitive trivalent graphs.
	\item 1981, Weiss: No s-arc-transitive graphs with $\val\geq 3$ for $s=6$ and $s\geq 8$.
	\item 2019, Li C.H. \& Xia B.Z.: \\ Connected \textbf{non-bipartite} 3-arc-transitive solvable Cayley graph with $\val\geq 3$ is a normal cover of the Hoffman-Singleton graph or the Peterson graph.
	\item 2021, Zhou J.X.: No such normal covers, so $s\leq 2$ sharply.
\end{itemize}

\begin{problem}
Studying connected \textbf{bipartite} s-arc-transitive solvable Cayley graphs, and determining the upper bound on $s$.	
\end{problem}
For convenience, one may assume $s\geq 3$ and $\val\geq 3$. 

\end{frame}




\section{Main Result \& Examples}
\begin{frame}{Main Result}
\begin{theorem}
Every connected s-arc-transitive solvable Cayley graph with $s\geq 3$ and $val\geq 3$ is a normal cover of one of the following graphs:	
\begin{enumerate}
	\item the complete bipartite graph $K_{n,n}$ with $n\geq 3$;
	\item the geometry incidence graph $\mathcal{GI}(5, 2, 2)$;
	\item the standard double cover of the Hoffman-Singleton graph;
	\item a graph $\Sigma$ with valency $p^f+1$ such that $\PSL_3(p^f).2\leq \Aut\Sigma\leq \Aut(\PSL_3(p^f))$ and $\mathbb{Z}_p^{2f}:\SL_2(p^f)\triangleleft(\Aut\Sigma)_\alpha$, where $p^f\geq 3$ is a prime power and $\alpha$ is a vertex.
\end{enumerate}
In particular, the sharp upper bound on s is 4.
\end{theorem}
These normal covers are being investigated in a sequel. (usually challenge)
\end{frame}

\begin{frame}{Examples}
(1) $\mathbf{K}_{n,n}$ with $n\geq 3$
\begin{itemize}
	\item 3-transitive
	\item $\mathbf{K}_{n,n}\cong \Cay(G,G\backslash H)$ where $G$ is solvable and $H<G$ of index 2
\end{itemize}

(2) $\mathcal{GI}(5,2,2)$
\begin{itemize}
	\item the incidence graph of $(\mathcal{P,L})$ where $\mathcal{P}$(resp. $\mathcal{L}$) is the set of 2-subspace(resp. 3-subspace) of $\mathbb{F}_2^5$
	\item $\Aut(\GI(5,2,2))=\GL_5(2).\langle \sigma\rangle$ is vertex-transitive
	\item $G_\alpha=2^6:(\GL_2(2)\times\GL_3(2))$, $G_{\alpha\beta}=2^8:(S_3\times S_3)$, val=$\frac{|G_\alpha|}{|G_{\alpha\beta}|}$=7
	\item 3-transitive\footcite{LI20162907}$^\text{,Theorem 3.4}$
	\item $G=RG_\alpha$ with $R\cong 31:5:2$ vertex-regular\footcite{LI2022factorizations}$^\text{, Theorem 1.1}$
\end{itemize}

\end{frame}

\begin{frame}{Examples}
(3) $\HS_{50}^{(2)}$
\begin{itemize}
	\item \textbf{standard double cover:} $\Gamma^{(2)}=(\tilde{V},\tilde{E})$ of $\Gamma=(V,E)$, where\\ $\tilde{V}=V\times\{1,2\}$, $\tilde{E}=\{\{(v,1),(w,2)\}\mid\{v,w\}\in E\}$
\end{itemize}
\end{frame}

\begin{frame}{Examples}
(4) $\mathcal{PH}(3,q)$
\begin{itemize}
	\item the incidence graph of $\PG(2,q)=(\mathcal{P,L})$, where \\$\mathcal{P}$(resp. $\mathcal{L}$) is the set of 1-subspace(resp. 2-subspace) of $\mathbb{F}_q^3$
	\item $\Aut(\PH(3,q))=\PGaL_3(q).\langle \sigma\rangle=\Aut(\PSL_3(q))$
	\item $(\alpha_0,\cdots,\alpha_4)$ with $\alpha_0\in \mathcal{P}$ corresponds to a basis $v_1,v_2,v_3$ such that\[ \alpha_0=\langle v_1\rangle, \alpha_1=\langle v_1,v_2\rangle, \alpha_3=\langle v_2\rangle, \alpha_4=\langle v_2,v_3\rangle, \alpha_5=\langle v_3\rangle \]all ordered bases are equiv. under linear transformation\\$\implies$ 4-arc-transitive
	\item a Cayley graph\footcite{MARUSIC2003162} of $D_{2(q^2+q+1)}$ 
\end{itemize}
\end{frame}


\section{Proof of the Main result}
\subsection{Overview of the proof}
\subsection{Reduct to almost simple groups}
\subsection{As case}


%%%%%%%%%%%%%%%%%%%%%%%%%%%%%%%



\begin{frame}{References}
\small
\printbibliography
\end{frame}



\end{document}


