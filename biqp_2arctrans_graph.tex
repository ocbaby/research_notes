\documentclass[a4,11pt]{article}

\usepackage[utf8]{inputenc}
\usepackage{amsmath}
\usepackage{amsfonts}
\usepackage{amssymb}
\usepackage{amsthm}
\usepackage{graphicx}


\newtheorem{theorem}{Theorem}[theorem]
\newtheorem{corollary}{Corollary}[theorem]
\newtheorem{proposition}{Proposition}[theorem]
\newtheorem{lemma}{Lemma}[theorem]
\newtheorem{conjecture}{Conjecture}[theorem]
\newtheorem{definition}{Definition}[theorem]
\newtheorem{example}{Example}[theorem]
\newtheorem{remark}{Remark}[theorem]

\title{Bi-quasiprimitive 2-arc-transitive graphs}
\author{Yuandong Li}
\date{\today}


\begin{document}

\maketitle

\tableofcontents

\section{Introduction}
My PhD thesis may focus on this topic. 
Do some basic work on bi-quasiprimitive actions and on 2-arc-transitive graphs.

\section{Preliminaries}
\subsection{O\'Nan-Scott-Praeger Thoerem}

\section{Bi-quasiprimitive actions}

\section{Bi-partite 2-arc-transitive graphs}

\end{document}
