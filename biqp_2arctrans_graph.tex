\documentclass[a4,11pt]{article}

\usepackage[utf8]{inputenc}
\usepackage{amsmath}
\usepackage{amsfonts}
\usepackage{amssymb}
\usepackage{amsthm}
\usepackage{graphicx}


\newtheorem{theorem}{Theorem}[theorem]
\newtheorem{corollary}{Corollary}[theorem]
\newtheorem{proposition}{Proposition}[theorem]
\newtheorem{lemma}{Lemma}[theorem]
\newtheorem{conjecture}{Conjecture}[theorem]
\newtheorem{definition}{Definition}[theorem]
\newtheorem{example}{Example}[theorem]
\newtheorem{remark}{Remark}[theorem]

\title{Bi-quasiprimitive 2-arc-transitive graphs}
\author{Yuandong Li}
\date{\today}


\begin{document}

\maketitle

\tableofcontents

\section{Introduction}
My PhD thesis may focus on this topic. 
Do some basic work on bi-quasiprimitive actions and on 2-arc-transitive graphs.

The following are generated by chatgpt.

Q:Is there a theory in graph theory works just like Galois theory, which is construct a correspondence between field extension and normal subgroups.
A:Yes, there is a theory in graph theory that is analogous to Galois theory, and it is known as "Galois correspondence in graph theory" or "Galois theory of graphs".

In Galois theory, the Galois correspondence establishes a one-to-one correspondence between the intermediate fields of a field extension and the normal subgroups of its Galois group. Similarly, in the Galois theory of graphs, there is a correspondence between the subgroups of the automorphism group of a graph and the so-called "coverings" of the graph.

A covering of a graph is a graph that is obtained by "lifting" the original graph to a higher-dimensional space, so that the vertices of the covering graph correspond to the orbits of the vertices of the original graph under the action of the automorphism group. The covering graph then encodes information about the symmetries of the original graph.

The Galois correspondence in graph theory establishes a one-to-one correspondence between the subgroups of the automorphism group of a graph and the coverings of the graph. Just as in Galois theory, this correspondence can be used to study the symmetries of a graph and to understand the relationships between different coverings of the same graph.



Q:Is there a problem in graph theory whose role is similar to the problem about radical solution of polynomial in Galois theory?
A:Yes, there is a problem in graph theory that is analogous to the problem of finding the radical solution of a polynomial in Galois theory, and it is known as the "isomorphism problem" for graphs.

The isomorphism problem for graphs asks whether two given graphs are isomorphic, i.e., whether there exists a one-to-one correspondence between their vertices that preserves the adjacency relation.

This problem is similar to the problem of finding the radical solution of a polynomial in Galois theory, because just as the radical of a polynomial is the smallest radical extension of the field containing all the roots of the polynomial, the isomorphism problem asks whether there exists a "smallest" graph that contains all the information about the isomorphism class of a given graph.

In fact, just as the Galois group of a polynomial acts on its roots, the automorphism group of a graph acts on its vertices and edges, and this action can be used to study the symmetries of the graph.

The isomorphism problem is known to be in the class of problems known as NP (nondeterministic polynomial time), which means that it is believed to be computationally difficult, but not necessarily impossible, to solve efficiently for large graphs. Many algorithms have been developed to tackle the isomorphism problem, but there is still no known algorithm that works efficiently for all graphs.

\section{Preliminaries}
\subsection{O\'Nan-Scott-Praeger Thoerem}

\section{Bi-quasiprimitive actions}

\section{Bi-partite 2-arc-transitive graphs}

\end{document}
