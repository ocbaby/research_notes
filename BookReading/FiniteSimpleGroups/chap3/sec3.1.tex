% '\def\SubfileName{}' is needed before include or input this file
% Change 'SubfileName' to the name of the subfile.

\ifx\ChapThreeSecOne\undefined % Compile separately
    \documentclass[a4paper & 11pt]{article}
    

\usepackage[utf8]{inputenc}
\usepackage{amsmath}
\usepackage{amsfonts}
\usepackage{amssymb}
\usepackage{amsthm}
\usepackage{graphicx}
\usepackage{enumerate}
\usepackage{geometry}
\usepackage{hyperref}
\usepackage{xcolor}
\usepackage{indentfirst}
\usepackage{enumitem}


\geometry{
  paper=a4paper,
  margin=64pt,
  %includeheadfoot
}
\linespread{1.3}
\setlength{\parskip}{3pt}





\newtheorem{theorem}{Theorem}[section]
\newtheorem{corollary}[theorem]{Corollary}
\newtheorem{proposition}[theorem]{Proposition}
\newtheorem{lemma}[theorem]{Lemma}
\newtheorem{conjecture}[theorem]{Conjecture}
\newtheorem{definition}[theorem]{Definition}
\newtheorem{example}[theorem]{Example}
\newtheorem{remark}[theorem]{Remark}




\def\SL{\mathrm{SL}}
\def\GL{\mathrm{GL}}
\def\PSL{\mathrm{PSL}}
\def\PGL{\mathrm{PGL}}
\def\PSU{\mathrm{PSU}}
\def\PSp{\mathrm{PSp}}
\def\PO{\mathrm{P\Omega}}
\def\Aut{\mathrm{Aut}}
\def\Inn{\mathrm{Inn}}
\def\Out{\mathrm{Out}}
\def\Cay{\mathrm{Cay}}

 % common config

    % additional config for this file
    \def\maintitle#1{\section*{#1}}
    \def\subtitle#1{\section{#1}}

    \begin{document}

\else % Compile as subfile
    \ifx\chaptitle\undefined % Compile whole book
        \def\maintitle#1{\subsection{#1}}
        \def\subtitle#1{\subsubsection{#1}}
    \else % Compile chapter
        \def\maintitle#1{\section{#1}}
        \def\subtitle#1{\subsection{#1}}
    \fi
\fi
%%%%%%%%%%%%%%%%%%%%%%%

\maintitle{Introduction}

'Classical' simple groups: linear groups, unitary groups,  symplectic groups, orthogonal groups.

Mainly obtained by taking $G'/Z(G')$ from suitible matrix groups $G$.


\begin{table}[htbp]
    \centering
    \begin{tabular}{lllll}
        Definition &
        Simplicity & 
        Subgroups & 
        Automorphisms \& Covering groups & 
        Isomorphisms \\
        \hline
        $PSL_n(q)$ &
        Iwasawa's lemma &
        geometry &
        (briefly mentioned) &
        projective spaces \\
    \end{tabular}
\end{table}

Symplectic groups: easy to understand, orders, simplicity, subgroups, covering groups, automorphisms, generic isomorphism $\mathrm{Sp}_2(q)\cong\mathrm{SL}_2(q)$, exceptional isomorphism $\mathrm{Sp}_4(2)\cong S_6$.

Unitary groups: similar to symplectic groups.

Orthogonal groups:\vspace{-3mm}
\begin{itemize}[itemsep=0pt]
    \item fundamental differences between the cases of $\mathrm{char}{F}=2$ or odd
    \item subquotient is not usually simple
    \item to get usually simple groups, using spinor norm for odd char (see Clifford algebras and spin groups), and quasideterminant for char 2
    \item generic isomorphisms $\PO_6^+(q)\cong \PSL_4(q)$, $\PO_6^-(q)\cong\PSU_4(q)$, $\PO_5(q)\cong\PSp_4(q)$ all derive from the Klein correspondence
\end{itemize}

A simple version of Aschbacher-Dynkin theorem is proved, relying heavily on representation theory.

More explicit versions for individual classes of groups see Kleidman and Liebeck's book.

Some exceptional behavior of small classical groups is related to exceptional Weyl groups.


%%%%%%%%%%%%%%%%%%%%%%%
\ifx\ChapThreeSecOne\undefined % Compile separately
     %\bibliography{abcd}
     \end{document}
\fi
