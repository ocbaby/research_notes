% '\def\SubfileName{}' is needed before include or input this file
% Change 'SubfileName' to the name of the subfile.

\ifx\ChapThreeSecThree\undefined % Compile separately
    \documentclass[a4paper,11pt]{article}
    

\usepackage[utf8]{inputenc}
\usepackage{amsmath}
\usepackage{amsfonts}
\usepackage{amssymb}
\usepackage{amsthm}
\usepackage{graphicx}
\usepackage{enumerate}
\usepackage{geometry}
\usepackage{hyperref}
\usepackage{xcolor}
\usepackage{indentfirst}
\usepackage{enumitem}


\geometry{
  paper=a4paper,
  margin=64pt,
  %includeheadfoot
}
\linespread{1.3}
\setlength{\parskip}{3pt}





\newtheorem{theorem}{Theorem}[section]
\newtheorem{corollary}[theorem]{Corollary}
\newtheorem{proposition}[theorem]{Proposition}
\newtheorem{lemma}[theorem]{Lemma}
\newtheorem{conjecture}[theorem]{Conjecture}
\newtheorem{definition}[theorem]{Definition}
\newtheorem{example}[theorem]{Example}
\newtheorem{remark}[theorem]{Remark}




\def\SL{\mathrm{SL}}
\def\GL{\mathrm{GL}}
\def\PSL{\mathrm{PSL}}
\def\PGL{\mathrm{PGL}}
\def\PSU{\mathrm{PSU}}
\def\PSp{\mathrm{PSp}}
\def\PO{\mathrm{P\Omega}}
\def\Aut{\mathrm{Aut}}
\def\Inn{\mathrm{Inn}}
\def\Out{\mathrm{Out}}
\def\Cay{\mathrm{Cay}}

 % common config

    % additional config for this file
    \def\maintitle#1{\section*{#1}}
    \def\subtitle#1{\section{#1}}

    \begin{document}

\else % Compile as subfile
    \ifx\chaptitle\undefined % Compile whole book
        \def\maintitle#1{\subsection{#1}}
        \def\subtitle#1{\subsubsection{#1}}
    \else % Compile chapter
        \def\maintitle#1{\section{#1}}
        \def\subtitle#1{\subsection{#1}}
    \fi
\fi
%%%%%%%%%%%%%%%%%%%%%%%

\maintitle{Linear groups}

Generally speaking, the classification of a certain kind of algebraic objects goes through four steps: extracting abstract concept from various examples, accumulating natural and classical families, organizing by analysis on generic properties and finally collecting sporadic cases. 

As for finite simple groups, the motivation comes from Jordan-Holder theorem, since which simple groups are deemed as elementary bricks. The families of cyclic groups and alternating groups gives the very first examples. After that, mathematicians find that there are many finite simple groups of Lie type, which stem from the study of Lie algebras. Actually, such groups forms a quite large family which turns out to be the main part of the classification and is divided into classical and exceptional parts during processing. The sporadic groups are the last part, which are found case-by-case.

In this chapter, we will introduce the family of linear groups, which is the basic case of groups of Lie type, since others can be seem as stabilizers of certain structures on vector spaces.

\subtitle{Introduction}

% definition, order, action, corelation

The story begins with the automorphisms of linear spaces, similar to the case of symmetric groups on sets.
\begin{definition}
    The so called \textbf{general linear group} $\GL(n,q)$ is the group of  all invertible linear transformations over vector space $V=\mathbb{F}_q^n$, or equivalently, all invertible $n\times n$ matrices over $\mathbb{F}_q$.
\end{definition}

Since linear group can be defined from two ways, algebraic (A) / geometric (G), there is also two parallel ways to deal with problems of linear groups. 
Here we follow the geometric way.

\begin{example}
    $\GL_2(2)\cong S_3$, $\GL_2(3)\cong 2.S_4$.
\end{example}

\begin{remark}
    $\GL_n(q)$ acts regularly on ordered basis of $V$. \\Thus
    $|\GL(n,q)|=(q^n-1)(q^n-q)\cdots(q^n-q^{n-1})=q^{n(n-1)/2}(q-1)\cdots(q^n-1)$.\\
    $(q-q^{-1}-q^{-2})q^{n^2} < |\GL(n,q)|\leq (1-q^{-1})(1-q^{-2})q^{n^2}$.
\end{remark}

\begin{corollary}
    $\AGL_n(q)$ acts 2-transitively on $\mathbb{F}_q^n$. $\AGL_n(2)$ acts 3-transitively on $\mathbb{F}_2^n$. 
\end{corollary}
\begin{proof}
    Note that for any two non-zero vectors $u,v\in\mathbb{F}_2^n$, $u,v$ are linear dependent iff. $u=v$. Thus $\GL_n(2)$ is 2-transitive on $V$.
\end{proof}

However, the simple groups do not come out from $\GL(n,q)$ directly. But we have some clues.

\begin{proposition}
    A finite non-abelian simple group is perfect and center-free.
\end{proposition}

Generally, $\GL_n(q)$ is neither perfect nor center-free. Since perfection is inherited when taking quotient, we try to do two things: taking derived subgroup till perfect and then moduling center till center-free. This could lead to some simple groups.

Firstly, we need to find the derived subgroup of $\GL_n(q)$. Note that commutators in $\GL_n(q)$ are of the form $[A,B]=ABA^{-1}B^{-1}$, which has determinant 1. Hence we can restrict our scope to a subgroup.

\begin{definition}
    Consider the group homomorphism $\mathrm{det}:\GL_n(q)\to \mathbb{F}_q^\times$, $g\mapsto \lambda_1\cdots\lambda_n$, its kernel is denoted as $\SL_n(q)$, named \textbf{special linear group}.
\end{definition}

\begin{remark}
    $|\SL_n(q)|=|\GL_n(q)|/|\mathbb{F}_q^\times|=q^{n(n-1)/2}(q^2-1)\cdots(q^n-1)$.
\end{remark}

Similar as $S_n$, $A_n$ has basic generators 2-cycles, 3-cycles resp. with the most fixed points, we look at a $\tau\in \GL(V)$ which fixes a hyperplane $W$ point-wise. Suppose $V=W\oplus\langle v\rangle$. Then 

\begin{definition}
\begin{equation*}
    v^\tau=\begin{cases}
    \alpha v \text{ for }\alpha\in F\backslash\{0,1\}, & \tau\text{ is called a \textbf{dilatation} or a \textbf{homology} (in projective version).}\\
    v+w \text{ for } w\in W\backslash \{0\}, & \tau\text{ is called a \textbf{transvection} or an \textbf{elation} (in projective version).}\\
    \alpha v+w \text{ for }\alpha\in F\backslash\{0,1\}, w\in W, & u:=v+(a-1)^{-1}w \text{ then reduce to the first case.}
    \end{cases}
\end{equation*}
Formally, a transvection $\tau(w,\varphi)\in\GL(V)$ is a map of the form:
\[ v\mapsto v+\varphi(v)w, \forall v\in V \] where $\varphi\in V^*\backslash \{0\}$ and $w\in\ker \varphi$.
\end{definition}

\begin{lemma}[properties of transvections]\ 
    \begin{enumerate}
        \item $\tau\in \SL$;
        \item $\tau(\alpha w,\varphi)=\tau(w,\alpha\varphi)$, $\tau(w,\varphi)^{-1}=\tau(-w,\varphi)=\tau(w,-\varphi) $;
        \item $\tau(w_1,\varphi)\tau(w_2,\varphi)=\tau(w_1+w_2,\varphi)$;
        \item $\tau(w,\varphi_1)\tau(w,\varphi_2)=\tau(w,\varphi_1+\varphi_2)$;
        \item $(\tau(w,\varphi))^g=\tau(w^g,\varphi\circ g)$, $\forall g\in \GL$;
        \item all transvections are conjugate in $\GL_{n\geq 2}(q)$ and $\SL_{n\geq 3}(q)$ by adjusting images in $\ker\varphi_2\backslash\langle w_2\rangle$.
        \item any two independent vectors / distinct hyperplanes are equivalent under transvections.
    \end{enumerate}
\end{lemma}

\begin{lemma}\label{abelian normal subgroup}
    $T_w:=\{\tau(w,\varphi)\mid \varphi\in V^*, \varphi(w)=0\}$ is an abelian normal subgroup of $(\SL_n(q))_w$.
\end{lemma}

\begin{lemma}\label{generating}\ 
    \begin{enumerate}[itemsep=0pt,label=\roman*.]
        \item The transvections (elations) generate $\SL$ $(\PSL)$.
        \item The transvections (elations) together with dilatations (homologies) generate $\GL$ $(\PGL)$.
    \end{enumerate}
\end{lemma}

\begin{proof}
    Let $T$ be the group generated by transvections. Obviously, $T\leq \SL$. \\As for the other direction we do induction on $n$.\\If $n=1$, then $T=1=\SL_1(q)$. Suppose $n\geq 2$ and $V=W\oplus\langle v\rangle$. \\
    Then $\forall \rho\in\SL_n(q)$, $\exists \tau_1\in T$ maps $v^{\rho}$ to $v=v^{\rho}+(v-v^{\rho})\not\in  W^{\rho\tau_1}\cup W$. \\
    And further $\exists \tau_2$ fixing $W^{\rho\tau_1}\cap W+\langle v^{\rho \tau_1}\rangle$ and taking $W^{\rho \tau_1}$ to $W$.\\
    Now $(\rho\tau_1\tau_2)|_W\in\SL(W)$ hence is a product of transvections on $W$.\\
    Expanding them to transvections on $V$ we can express $\rho$ as product of transvections.
\end{proof}

\begin{lemma}\label{perfectness}
    $\GL_n(q)'=\SL_n(q)=\SL_n(q)'$ except for $\SL_2(2)\cong S_3$, $\SL_2(3)'\cong Q_8$.
\end{lemma}
\begin{proof}
    Since $\SL'\leq \GL'\leq \SL$, we only need to prove $\SL_n(q)\leq \SL_n(q)'$. \\Since transvections are conjugate in $\GL_n(q)$ and $\SL'_n(q)\ \mathrm{char}\ \SL_n(q)\trianglelefteq \GL_n(q)$, \\
    it is sufficient to show that there is a transvection being a commutator in $\SL_n(q)$.  \\
    For $n\geq 3$, take $g\in\SL_n(q)$ and $0\neq w\neq w^g$, \[\tau(w^g-w,\varphi)=\tau(-w,\varphi)g^{-1}\tau(w,\varphi)g=[\tau(w,\varphi),g]\]
    For $n=2$ and $q\geq 4$, take $V=\langle u,v\rangle$, $\tau:u\mapsto u$, $v\mapsto u+v$, $g=\mathrm{diag}(a,a^{-1})$, $a\in F\backslash\{0,1\}$. \\
    Then $\tau((1-a^2)u,-\varphi)=[\tau(u,\varphi),g]$.\\
    Exceptions: $\GL_2(2)=\SL_2(2)$, $\GL_2(3)\cong 2.S_4\cong Q_8:S_3$.
\end{proof}

Now we consider the center. By linear algebra, $Z:=Z(\GL_n(q))$ consists of all scalar matrices and isomorphic to $\mathbb{F}_q^\times$. (Actually $Z(\SL_n(q))\leq Z$ for the same reason by considering $C_{\GL_n(q)}(\{ I+E_{ij}|i\neq j\})$. Although, $Z\cap\SL_n(q)$ is already a normal subgroup which we want to quotient out.) By taking quotient we get \textbf{projective general linear groups} $\PGL_n(q):=\GL_n(q)/Z$ and \textbf{projective special linear groups} $\PSL_n(q):=\SL_n(q)/(Z\cap\SL_n(q))$. By definition, $\PSL_n(q)$ is not a subgroup but is isomorphic to a normal subgroup of $\PGL_n(q)$.

\begin{remark}
    $|\PGL_n(q)|=|\GL_n(q)|/|\mathbb{F}_q^\times|=q^{n(n-1)/2}(q^2-1)\cdots(q^n-1)$.

    $|\PSL_n(q)|=|\SL_n(q)|/|Z\cap \SL_n(q)|=\frac{1}{(n,q-1)}q^{n(n-1)/2}(q^2-1)\cdots(q^n-1)$.
\end{remark}

\begin{remark}
    For $(n,q)\neq (2,2),(2,3)$, $\SL$ is perfect and hence a covering group of $\PSL$. However, if $q-1>(n,q-1)$, then $Z\not\subseteq \GL'\leq \SL$ and hence $\GL$ is not a covering group of $\PGL$.
\end{remark}

\begin{proposition}
    \[\SL\cong \PGL\iff \SL\cap Z=1 \iff (n,q-1)=1,\quad\SL=\GL(\cong\PGL\cong\PSL) \iff q=2.\]
\end{proposition}

Now we introduce some actions of linear groups.

\begin{definition}
    The \textbf{projective geometry} of $V=\mathbb{F}_q^n$ is the set of all 1-dimensional subspaces of $V$, denoted as $\PG(n-1,q)$.
\end{definition}

\begin{proposition}
    $\GL_n(q)$ acts transitively on $\PG(n-1,q)$ with kernel $Z(\GL_n(q))$. Thus $\PGL_n(q)$ acts faithfully transitively on $\PG(n-1,q)$. 
\end{proposition}


\begin{proposition}
    $\PGL_n(q)$ acts regularly on \textbf{frames} of $\PG(n-1,q)$, the set of all $(n+1)$-tuples on $\PG(n-1,q)$ with the property that no $n$ points lie in a hyperplane (each n points form a basis).
\end{proposition}
\begin{proof}
    $\GL_n(q)$ acts transitively on $(\langle e_1\rangle,\cdots,\langle e_n\rangle,\langle \sum_{i=1}^n \alpha_i e_i\rangle)$ with stabilizer scalar matrices.
\end{proof}

\begin{proposition}
    $\PGL_2(q)$ is sharply 3-transitive on $\PG(1,q)$, while $\PGL_{n>2}(q)$ is only 2-transitive on $\PG(n-1,q)$. 
\end{proposition}
\begin{proof}
    Any three distinct points in $\PG(1,q)$ form a frame. \\However, three distinct points in $\PG(n-1, q)$ with $n > 2$ might be collinear or not.
\end{proof}

\begin{corollary}\label{primitive action}
    $\PSL_n(q)$ acts 2-transitively on $\PG(n-1,q)$ by suitibly choosing images to adjust the determinant to be 1.
\end{corollary}

\begin{remark}
    [Explict action of $\PGL_2(q)$ on $\PG(1,q)$ by \textbf{linear fractional representation}]
    Let $z:=\langle (z,1)^T\rangle$, $\infty:=\langle (1,0)^T\rangle$, then $\PG(1,q)=\{1,2,\cdots,q-1,\infty\}$ and
    \begin{equation*}
        \bar{g}\in\PGL_n(q): z\mapsto \frac{az+b}{cz+d}, \quad \forall g=
        \begin{pmatrix}
            a & b\\
            c & d
        \end{pmatrix}\in \GL_2(q) 
    \end{equation*}
\end{remark}


\begin{theorem}[Fundamental Theorem of Projective Geometry]
    $\Aut(\PG(n-1,q))=\PGammaL(n,q)$. 
\end{theorem}



\subtitle{Simplicity of $\PSL_n(q)$}


\begin{lemma}[Iwasawa]
    If finite group $G$ satisfies the following conditions, then $G$ is simple.
    \begin{enumerate}[itemsep=0pt,label=\roman*.]
        \item $G'=G$;
        \item $G$ is primitive on some set $\Omega$;
        \item $\exists A\trianglelefteq G_\alpha$ where $A$ is solvable;
        \item $G=A^G$.
    \end{enumerate}
    i.e. A perfect primitive group G, being the normal
    closure of an abelian normal subgroup A of its point stabilizer, is simple.
\end{lemma}
\begin{proof}
    Suppose that $1\neq N\trianglelefteq G$. Then, by primitivity, $N$ is transitive on $\Omega$ and hence $G=G_\alpha N$. For any $g\in G$, $g=hn$ for some $h\in G_\alpha$ and $n\in N$. 
    
    Then $a^g=a^{hn}=a^n$, $\forall a\in A$, since $A\trianglelefteq G_\alpha$. Moreover, $a^n=a(n^{-1})^{a}n\in AN$ since $N\trianglelefteq G$. Thus $G=A^G= AN$.
    
    Now, $G/N=AN/N=A/(A\cap N)$ is solvable. Meanwhile, $(G/N)'=G'N/N=GN/N=G/N$. Thus $G/N=1$ and $G=N$, $G$ is simple.
\end{proof}

\begin{theorem}
    $\PSL_n(q)$ is a simple group except for $\PSL_2(2)$ and $\PSL_2(3)$.
\end{theorem}

The proof proceeds along Iwasawa's lemma. We have check the four conditions.
\begin{enumerate}[itemsep=0pt,label=\roman*.]
    \item Find a primitive action of $G$; \ref{primitive action}
    \item Prove perfectness; \ref{perfectness}
    \item Find a solvable normal subgroup $A$ of point stabilizer; \ref{abelian normal subgroup}
    \item Prove $G=A^G$. \ref{generating}
\end{enumerate}



\subtitle{Sporadic behaviours}


\noindent\textbf{Isomorphism relations:}  
\begin{table}[htbp]
    \centering
    \begin{tabular}{lrl}
        Order 6 & $\PSL_2(2)$&$\cong S_3$\\
        Order 12 & $\PSL_2(3)$&$\cong A_4$\\
        Order 60 & $\PSL_2(4)\cong\PSL_2(5)$ &$\cong A_5$\qquad $\PGL_2(5)\cong S_5$\\
        Order 168 & $\PSL_2(7)$ &$\cong PSL_3(2)$\\
        Order 360 & $\PSL_2(8)$ &$\cong A_6$\\
        Order 20160 & $\PSL_3(4)\not\cong\PSL_4(2)$ &$\cong A_{8}$\\
    \end{tabular}
\end{table}

\begin{theorem}
    For prime $p > 3$, a simple group of order $(p - 1)p(p + 1)/2$ is isomorphic to $\PSL_2(p)$.
\end{theorem}

\begin{proposition}
    There are infinitely many pairs of non-isomorphic simple
groups of the same order ($B_n(q) \not\cong C_n(q)$, $\PSL_3(4) \not\cong A_8$), however, no such triples.
\end{proposition}



\noindent\textbf{Exceptional permutation representations:}
In genaral, the action of
$\PSL_2(q)$ on $\PG(1,q)$ of degree $q + 1$ is the smallest permutation representation, except:
\begin{itemize}
    \item $\PSL_2(5)\cong A_5$ acts on 5 points ($\PGL_2(5)$ as stabilizer of $S_6$)
    \item $\PSL_2(7)\cong \PSL_3(2)$ acts on $|\PG(2,2)|=7$ projective points(lines)
    \item $\PSL_2(11)$ acts on the 11 images of the partition $(\infty 0|12|36|48|5X|79)$ of $\PG(1, 11)$
\end{itemize}




\noindent\textbf{Exceptional covers:}
In general, $\SL_n(q)$ is the full covering group of simple $\PSL_n(q)$, except for :
\begin{itemize}
    \item $\PSL_2(4)$, $\PSL_3(2)$, $\PSL_4(2)$ have an exceptional double cover
    \item $\PSL_2(9)\cong A_6$ has an exceptional triple cover
    \item $\PSL_3(4)$ has an exceptional cover as $4^2.\PSL_3(4)$
\end{itemize}


\subtitle{Outer automorphisms}

\textit{`It is a fact that the outer automorphism groups of all the classical groups have a uniform description in terms of so-called \textbf{diagonal, field, and graph} automorphisms.’}

\begin{itemize}
    \item Diagonal automorphism $\delta$: $X\mapsto \Lambda^{-1} X\Lambda$ w.r.t. dilatation $\Lambda=\mathrm{diag}(\lambda,1,\cdots,1)$, $\mathbb{F}_q^\times=\langle \lambda\rangle$.
        \subitem- $\GL_n(q)=\SL_n(q):\langle \Lambda\rangle$ acts by conjugation on $\SL_n(q)$ with kernel scalar matrices
        \subitem- The action induces $\langle\delta\rangle\leq \Out(\SL_n(q))$ where $|\delta|=\frac{|\GL_n(q)|}{(q-1)|\Inn \SL_n(q)|}=(n,q-1)$
        % \subitem- $\Lambda^{q-1}=I_n$, $\Lambda^n=\mathrm{diag}(\lambda^{n-1},\lambda^{-1},\cdots,\lambda^{-1})\mathrm{diag}(\lambda,\lambda,\cdots,\lambda)\in\SL_n(q)Z(\GL_n(q))$
        \subitem- $\PGL_n(q)=\PSL_n(q).\langle \delta\rangle$ 
    \item Field automorphism $\phi$: $(a_{ij})\mapsto (a_{ij}^p)$ w.r.t. some basis
        \subitem- $|\phi|=e$ where $q=p^e$
        \subitem- $\GammaL=\GL:\langle \phi\rangle$, $\SigmaL=\SL:\langle \phi\rangle$
        \subitem- each $(g,\phi^i)\in\GammaL$ corresponding to a semilinear map $v\mapsto (v^g)^{\phi^i}$
        \subitem- $\phi$ comes from linear groups of higher dimension: $\GammaL_n(p^e)\curvearrowright\mathbb{F}_p^{en}$ linearly ($a^p\equiv a\in\mathbb{F}_p$)
    \item Graph automorphism $\gamma$: $M\mapsto (M^T)^{-1}$ w.r.t. some pair of dual basis
        \subitem- inner automorphism when $n=2$, outer when $n\geq 3$
        \subitem- from higher dimension: $\GL_n(q).\langle\gamma\rangle\lesssim \GL_{2n}(q)$, $X\mapsto\mathrm{diag}(X^{-1},X^T)$, $\gamma\mapsto\widetilde{\left(\begin{smallmatrix}
            0 & I\\  I & 0
        \end{smallmatrix}\right)}$
\end{itemize}

Actually, these are the all outer automorphisms of classical groups.

\begin{theorem}\ 
    \begin{itemize}
        \item $\Out(\PSL_2(p^e))=\langle \delta\rangle\times \langle \phi\rangle\cong \mathbb{Z}_{(2,p^e-1)}\times\mathbb{Z}_e$ 
        \item $\Out(\PSL_n(p^e))=\langle \delta\rangle:(\langle \phi\rangle\times \langle\gamma\rangle)\cong \mathbb{Z}_{(n,p^e-1)}:(\mathbb{Z}_e\times\mathbb{Z}_2)\cong D_{2(n,p^e-1)}\times \mathbb{Z}_e$ for $n\geq 3$.
    \end{itemize}
\end{theorem}


\begin{example}
    $A_6\cong \PSL_2(9)$ has outer automorphism group $\langle\delta\rangle:\langle\phi\rangle\cong\mathbb{Z}_2\times\mathbb{Z}_2$.
    \[ \PSL_2(9).\langle\delta\rangle=\PGL_2(9),\quad \PSL_2(9).\langle\phi\rangle=\PSigmaL_2(9)\cong S_6,\quad \PSL_2(9).\langle\delta\phi\rangle=M_{10} \]
    Remark that $\PGL_2(9)\not\cong S_6$ since all order-3-elements in $\PSL_2(9)$ are conjugate while $S_6$ not.
\end{example}

\begin{definition} A classical group is one lying in the following chain:
    \begin{table*}[htbp]
        \centering
        \begin{tabular}{ccccccccccccc}
            &$\Omega (\bar{\Omega}) $&$\leq$&$ S (\bar{S}) $&$\leq $&$G (\bar{G})$&$\leq $&$C (\bar{C})$&$\leq $&$\Gamma (\bar{\Gamma})$&$\leq $&$A (\bar{A})$\\
            &Basic & & Special & & General & & Conformal & & Semilinear & & Automorphic \\ \midrule
            & $\SL$ & & $\SL$ & & $\GL$ & & $\GL$ & & $\GammaL$ & & $\Aut(\SL)$ \\
            &$\PSL$ & & $\PSL$ & & $\PGL$ & & $\PGL$ & & $\PGammaL$ & & $\Aut(\PSL)$ 
        \end{tabular}
    \end{table*}
\end{definition}
\ 

\newpage




\subtitle{Subgroups}

We start with the subgroups of $\GL_n(q)$.

\noindent\textbf{\underline{Subgroups from matrix:}}

\begin{table*}[htbp]
    \centering
    \begin{tabular}{llll}
        \toprule
        Grp. & Mat. & Name & Remark\\
        \midrule
        T & diagonal  & \textbf{maximal split torus} &  $\GL_n(q)_{(\langle v_1\rangle,\cdots,\langle v_n\rangle)}$ $v_i$'s linear indep.\\
        N & monomial & - & $\GL_n(q)_{\{\langle v_1\rangle,\cdots,\langle v_n\rangle\}}$ $v_i$'s linear indep.\\
        W & permutation  & \textbf{Weyl group} & irrelevant with q\\
        U & lower-unitriangular & \textbf{unipotent subgroup} & a Sylow p-subgroup of $\GL_n(q)$ \\
        B & lower-triangular & \textbf{Borel subgroup} & max.-flag-stab. of $\GL_n(q)$ \\
        \bottomrule
    \end{tabular}
    % \caption{Subgroups from matrix}
    \label{matsubgp}
\end{table*}
\begin{itemize}[itemsep=0pt,label=-]
    \item $W\cong S_n$
    \item $N=T:W\cong \GL_1(q)\wr S_n$
    \item $T=B\cap N\cong \GL_1(q)^n$
    \item $B=N_{\GL_n(q)}(U)=U:T\cong q^{n(n-1)/2}:\GL_1(q)^n$
\end{itemize}

\begin{definition}
    A \textbf{'BN-pair'} (aka. \textbf{Tits system}) is a pair $(B,N)$ of subgroups of a group $G$ such that
    \begin{enumerate}[itemsep=0pt,label=(\roman*)]
        \item $G=\langle B,N\rangle$;
        \item $T:=B\cap N\triangleleft N$;
        \item $W:=N/T$ is generated by a set $S$ of involutions;
        \item $S$ satisfies: $s\not\subseteq N_G(B)$ and $sBw\subseteq BswB\cup BwB$, $\forall s\in S$, $\forall w\in W$.  
    \end{enumerate}
\end{definition}

The set S is uniquely determined by B and N, the pair $(W,S)$ is a Coxeter system, $|S|$ is called the \textbf{rank}.

\begin{example}
    2-transitive actions $\sim$ BN-pairs of rank 1.
\end{example}

Every finte group of Lie type has a BN-pair, which allows one to give uniform, instead of case-by-case, proofs of many results. For example,
\begin{theorem}[Tits's simplicity theorem]
    If G has a BN-pair such that 
    \begin{enumerate}[itemsep=0pt,label=(\roman*)]
        \item B is a solvable group,
        \item the intersection of all conjugates of B is trivial,
        \item the set of generators of W cannot be decomposed into two non-empty commuting sets,
    \end{enumerate} then G is simple whenever it is a perfect group.
\end{theorem}

\noindent\textbf{\underline{Subgroups from geometry:}}

\begin{definition}
    \textbf{flag}: $0=V_0< V_1<\cdots<V_k=V$
\end{definition}

Borel subgroup B is a stabilizer of a \textbf{maximal flag} $0=V_0<\cdots <V_n=V$.

\begin{lemma}
    $\SL_n(q)$ is transitive on maximal flags.
\end{lemma}
\begin{proof}
    In general $\SL_n(q)$ may not be transitive on ordered basis, but must transitive on the ordered 1-spaces spanned by basis vectors. And maximal flag can be identified with such structure.
\end{proof}

\begin{definition}
    A \textbf{parabolic subgroup} $P$ is a stabilizer of a flag. (Alg: contain a Borel subgroup.)\\
    A \textbf{maximal parabolic subgroup} $P_k$ is a stabilizer of a \textbf{minimal flag} $0<W<V$ (a k-space).
\end{definition}

\begin{proposition}
    $P_k=Q:L\cong q^{k(n-k)}:(\GL_k(q)\times \GL_{n-k}(q))$ where \textbf{unipotent radical} $Q=\langle\left(\begin{smallmatrix}
        I & 0\\C & I
    \end{smallmatrix}\right)| C\in M_{n-k,k}\rangle$, \textbf{Levi-complement} $L=\langle\left(\begin{smallmatrix}
        A & 0\\0 & D
    \end{smallmatrix}\right)| A\in \GL_k(q), D\in \GL_{n-k}(q)\rangle$.
\end{proposition}


\begin{proposition}
    A maximal parabolic subgroup is a maximal subgroup.
\end{proposition}
\begin{proof}
    see Hua's notes page 24.
\end{proof}

\begin{table*}[htbp]
\centering
\begin{tabular}{cccc}
    \toprule
    Class & Subgroup & Structure & Remark\\
    \midrule
    $\mathscr{C}_1$ & parabolic & $q^{k(n-k)}:(\GL_k(q)\times \GL_{n-k}(q))$ & stab. of a k-space $W$\\
    $\mathscr{C}_2$ & imprimitive & $\GL_{n/m}(q)\wr S_m$ & stab. of $V=V_1\oplus\cdots\oplus V_m$\\
    $\mathscr{C}_3$ & superfield & $\GL_{n/r}(q).\mathbb{Z}_r$ & stab. of $\mathbb{F}_{q^r}/\mathbb{F}_q$\\
    $\mathscr{C}_4$ & tensor product & $\GL_k(q)\circ\GL_{n/k}(q)$ & stab. of $V=V_1\otimes V_2$\\
    $\mathscr{C}_5$ & subfield & $\GL_n(q_0)$ & stab. of a subfield $\mathbb{F}_{q_0}$\\
    $\mathscr{C}_6$ & symplectic-type & \makecell{$(\mathbb{Z}_{q-1}\circ r^{1+2a}).\Sp_{2a}(r)$\\ $(\mathbb{Z}_{q-1}\circ 2^{1+2a}).\Omega_{2a}^{\pm}(2)$}  & normalizers of symplectic-type r-groups\\
    $\mathscr{C}_7$ & \makecell{homogenous\\ tensor product} & $(\GL_r(q)\circ\cdots\circ\GL_r(q))\wr S_m$ & stab. of $V=V_1\otimes\cdots\otimes V_m$\\
    $\mathscr{C}_8$ & classical & $\Sp_n(q)$, $\GU_n(q^{1/2})$, $\GO_n^\epsilon(q)$ & stab. of non-degenerate forms\\
    $\mathscr{C}_9=\mathscr{S}$ & almost simple & $T\leq G\leq \Aut(T)$ & irreducible\\
    \bottomrule
\end{tabular}
\caption{Subgroup classes of classical groups}
\end{table*}

\noindent\textbf{\underline{Maximal subgroups:}}

Wilson's book provides a simple version of Aschbacher-Dynkin theorem for maximal subgroups of linear groups.

\begin{theorem}
    Any subgroup $H$ of $GL_n(q)$ not containing $SL_n(q)$ is contained in a subgroup of one of the following classes: $\mathscr{C}_1$, $\mathscr{C}_2$,  $\mathscr{C}_4$,   $\mathscr{C}_7$, $\mathscr{C}_6$, $\mathscr{C}_9$.
\end{theorem}
\begin{proof} Consider representation of $N=\mathrm{soc}(H)$ on $V$:\\
    NOT complete reducible $\implies$ $\mathscr{C}_1$\\
    complete reducible + NOT homogeneous $\implies$ $\mathscr{C}_1$ or $\mathscr{C}_2$(isom. comb. of homogeneous components)\\
    complete reducible + homogeneous + NOT irreducible $\implies$ $N\circ C_G(N)$ acts as  $\mathscr{C}_4$\\
    complete reducible + homogeneous + irreducible $\implies$ $N$ is unique minimal normal\\
    ...+ abelian $\implies$ $\mathscr{C}_6$\\
    ...+ non-abelian simple $\implies$ $\mathscr{C}_9$\\
    ...+ non-abelian non-simple $\implies$ $\mathscr{C}_7$
\end{proof}


Maximal subgroups of $\PGL$ can be easily obtained from those of $\GL$ by moduling scalar matrices. However, $M\lessdot \PGL\implies M\cap \PSL\lessdot \PSL$ may not hold in general, where $M$ is called \textbf{novel}. And conversely, $\SL$ may have maximal subgroups of classes $\mathscr{C}_3$, $\mathscr{C}_5$, $\mathscr{C}_8$.

An explicit list of maximal subgroups of $\SL_2(q)$: (see Hua's notes page 33)

Maximal subgroups of almost simple groups with socle $\PSL_2(q)$: (see Guidici,2007)



\begin{comment}
\subtitle{Applications}

role of AS in classification of 2-arc-trans graph: see [8,lemma 5.2-5.3] and [9]

classifying AS 2-arc transitive graphs (for certain infinite families of AS groups)

linear[Hassani], Suzuki or Ree [3,4], An/Sn (V-primitive) [10], 

\begin{example}
    Finite $G$ corefree $H<G$ and 2-element $g$ give a coset graph $\Gamma(G,H,HgH)$. It is $(G,2)$-arc-transitive iff cond1
\end{example}


If $\Gamma$ is $(G,2)$-arc-transitive where $G$ is 2-dim linear, then $\Gamma$ is a coset graph $\Gamma(G,H,HgH)$ above, i.e. no more. And a local discription.

\end{comment}




%%%%%%%%%%%%%%%%%%%%%%%
\ifx\ChapThreeSecThree\undefined % Compile separately
     %\bibliography{abcd}
     \end{document}
\fi
