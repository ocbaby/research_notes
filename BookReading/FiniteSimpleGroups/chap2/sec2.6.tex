% '\def\ChapTwoSecSix{}' is needed before include or input this file
\ifx\ChapTwoSecSix\undefined % Compile separately
    \documentclass[a4paper,11pt]{article}
    

\usepackage[utf8]{inputenc}
\usepackage{amsmath}
\usepackage{amsfonts}
\usepackage{amssymb}
\usepackage{amsthm}
\usepackage{graphicx}
\usepackage{enumerate}
\usepackage{geometry}
\usepackage{hyperref}
\usepackage{xcolor}
\usepackage{indentfirst}
\usepackage{enumitem}


\geometry{
  paper=a4paper,
  margin=64pt,
  %includeheadfoot
}
\linespread{1.3}
\setlength{\parskip}{3pt}





\newtheorem{theorem}{Theorem}[section]
\newtheorem{corollary}[theorem]{Corollary}
\newtheorem{proposition}[theorem]{Proposition}
\newtheorem{lemma}[theorem]{Lemma}
\newtheorem{conjecture}[theorem]{Conjecture}
\newtheorem{definition}[theorem]{Definition}
\newtheorem{example}[theorem]{Example}
\newtheorem{remark}[theorem]{Remark}




\def\SL{\mathrm{SL}}
\def\GL{\mathrm{GL}}
\def\PSL{\mathrm{PSL}}
\def\PGL{\mathrm{PGL}}
\def\PSU{\mathrm{PSU}}
\def\PSp{\mathrm{PSp}}
\def\PO{\mathrm{P\Omega}}
\def\Aut{\mathrm{Aut}}
\def\Inn{\mathrm{Inn}}
\def\Out{\mathrm{Out}}
\def\Cay{\mathrm{Cay}}


    \def\maintitle#1{\section*{#1}}
    \def\subtitle#1{\section{#1}}
    \begin{document}
\else % Compile as subfile
    \def\maintitle#1{\subsection{#1}}
    \def\subtitle#1{\subsubsection{#1}}
\fi
%%%%%%%%%%%%%%%%%%%%%%%


\maintitle{The O'Nan-Scott Theorem}
\subtitle{Some Lemmas}
\subtitle{The proof of the O'Nan-Scott Theorem}

Last week we introduced some lemmas and proved part of the O'Nan-Scott Theorem. This week we will finish the proof of the O'Nan-Scott Theorem.

\noindent\textbf{Notation:} Let $H$ be a subgroup of $S_n$ not containing $A_n$, $N$ be a minimal normal subgroup of $H$, and $K$ be the stabilizer in $H$ of a point.

$H$ intransitive $\implies$ case (i).

$H$ transitive imprimitive $\implies$ case (ii).

\noindent Now we assume $H$ primitive. And hence the discussion zoom into $\mathrm{soc}(H)$.

$\exists N$ abelian $\implies$ case (iv)affine.

\noindent Additionally we assume $\forall N$ nonabelian. 

If $H$ has more than one minimal normal subgroups $N_1\neq N_2$. 

It can be shown that $\exists x\in S_n$ conjugates $N_1$ to $N_2$. \textcolor{red}{specify x}

By corollary 2.11, $x$ also conjugates $N_2=C_H(N_1)$ to $N_1=C_H(N_2)$. \textcolor{red}{(Why?)} 

Hence $H<\langle H,x\rangle$, which has a unique minimal normal subgroup $N_1\times N_2$.

\noindent Additionally we assume $H$ has a unique minimal normal subgroup $N$, which is nonabelian.

$N$ simple $\implies C_H(N)=1\implies H\mathop{\curvearrowright}\limits^{\text{conj.}} N$ faithfully $\implies$ case (vi)AS.

$N=T^m=T_1\times\cdots\times T_m$ with $m>1$ $\implies$ $H\mathop{\curvearrowright}\limits^{\text{conj.}} \{T_1,\cdots,T_m\}$ transitively, and $K$ as well.

Let $K_i:=p_i(K\cap N)\leq T_i$ the projection of $K$ onto $T_i$. Then $K\cap N\leq K_1\times\cdots\times K_m$.



\noindent Case $K_i\neq T_i$ for some $i$: 

Now $K\cap N\leq K_1\times\cdots\times K_m < N$.

\underline{Claim:} $K$ normalizes $K_1\times\cdots\times K_m$. 

{\color{gray}
Since $K\cap N\triangleleft K$, $\forall k\in K$, $\forall x\in K\cap N$, 

we have $x=p_1(x)\cdots p_m(x)$, and $p_1(x)^k\cdots p_m(x)^k=x^k=p_1(x^k)\cdots p_m(x^k)\in K\cap N$. 

Then $p_i(x)^k=p_j(x^k)$ whenever $T_i^k=T_j$. (In direct product, equal iff. all coordinates equal.)

$\forall y \in K_1\times\cdots\times K_m$, $\exists x_1,\cdots, x_m\in K\cap N$ s.t. $y=p_1(x_1)\cdots p_m(x_m)$. 

Then $y^k=p_1(x_1)^k\cdots p_m(x_m)^k=p_1(x_{l_1}^k)\cdots p_m(x_{l_m}^k) \in K_1\times\cdots\times K_m$, where $T_i=T_{l_i}^k$.
}

By corollary 2.15, $K_1\times\cdots\times K_m=K\cap N$ and $K$ permutes $K_i$'s transitively. Let $k:=|T_i:K_i|$. 

Then $H=(T_1\times\cdots\times  T_m)\rtimes K\leq S_k\wr S_m\curvearrowright [T_1:K_1]\times\cdots\times[T_m:K_m]$ $\implies$ case (iii)PA.

\noindent Case $K_i= T_i$ for all $i$: 

\textit{Support} of $(t_1,\cdots,t_m)\in N$ is defined as $\{i\mid t_i\neq 1\}$.

$\Omega_1:=$ a non-empty min. supp. of an elt in $K\cap N$. $\implies$ $\Omega_1$ a block of $K,H\curvearrowright [m]$.

{\color{gray}
1 and all elts in $K\cap N$ with support $\Omega_1$ (i.e. $t_i\neq 1$ and $t_j=1$ $\forall i\in\Omega_1, \forall j\notin\Omega_1$) 

forms a normal subgp of $K\cap N$, which maps onto a normal subgp of hence $T_i$ itself $\forall i\in\Omega_1$.

$\Omega_1\cap \Omega_2\neq \emptyset\implies \exists x,y$ s.t. $[x,y]\neq 1$ has support contained in $\Omega_1\cap \Omega_2$, that is $\Omega_1$
}

$|\Omega_1|=1\implies N\leq K$, a contradiction.

$|\Omega_1|=m\implies$ $K\cap N=\{(t,\cdots,t)\mid t\in T\}$ WLOG.  $N\curvearrowright [N:K\cap N]$ $\implies$ case (v)diagonal.

\textcolor{gray}{$\forall i, \forall x, y \in K\cap N$, $p_i(x)=p_i(y)\implies p_i(xy^{-1})=1$ $\implies$ $xy^{-1}=1$ i.e. $p_i|_{K\cap N}$ inj. }


$|\Omega_1|=k\neq 1,m\implies N=\left(\mathop{\times}\limits_{i\in\Omega_1}T_i\right)^{l}\cong T^{kl}$, $N\cap K=\left(\mathrm{diag}\left(\mathop{\times}\limits_{i\in\Omega_1}T_i\right)\right)^l\cong T^l$.

The action of each $\mathop{\times}\limits_{i\in\Omega_1}T_i$ is diagonal of degree $r=|T|^{k-1}$. $H\leq S_r\wr S_l\curvearrowright [r]^l$ $\implies$ case (iii)PA.





%%%%%%%%%%%%%%%%%%%%%%%
\ifx\ChapTwoSecSix\undefined
     %\bibliography{abcd}
     \end{document}
\fi
