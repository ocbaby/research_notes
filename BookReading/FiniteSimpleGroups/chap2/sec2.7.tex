% '\def\SubfileName{}' is needed before include or input this file
% Change 'SubfileName' to the name of the subfile.

\ifx\ChapTwoSecSeven\undefined % Compile separately
    \documentclass[a4paper,11pt]{article}
    

\usepackage[utf8]{inputenc}
\usepackage{amsmath}
\usepackage{amsfonts}
\usepackage{amssymb}
\usepackage{amsthm}
\usepackage{graphicx}
\usepackage{enumerate}
\usepackage{geometry}
\usepackage{hyperref}
\usepackage{xcolor}
\usepackage{indentfirst}
\usepackage{enumitem}


\geometry{
  paper=a4paper,
  margin=64pt,
  %includeheadfoot
}
\linespread{1.3}
\setlength{\parskip}{3pt}





\newtheorem{theorem}{Theorem}[section]
\newtheorem{corollary}[theorem]{Corollary}
\newtheorem{proposition}[theorem]{Proposition}
\newtheorem{lemma}[theorem]{Lemma}
\newtheorem{conjecture}[theorem]{Conjecture}
\newtheorem{definition}[theorem]{Definition}
\newtheorem{example}[theorem]{Example}
\newtheorem{remark}[theorem]{Remark}




\def\SL{\mathrm{SL}}
\def\GL{\mathrm{GL}}
\def\PSL{\mathrm{PSL}}
\def\PGL{\mathrm{PGL}}
\def\PSU{\mathrm{PSU}}
\def\PSp{\mathrm{PSp}}
\def\PO{\mathrm{P\Omega}}
\def\Aut{\mathrm{Aut}}
\def\Inn{\mathrm{Inn}}
\def\Out{\mathrm{Out}}
\def\Cay{\mathrm{Cay}}

 % common config

    % additional config for this file
    \def\maintitle#1{\section*{#1}}
    \def\subtitle#1{\section{#1}}

    \begin{document}

\else % Compile as subfile
    \ifx\chaptitle\undefined % Compile whole book
        \def\maintitle#1{\subsection{#1}}
        \def\subtitle#1{\subsubsection{#1}}
    \else % Compile chapter
        \def\maintitle#1{\section{#1}}
        \def\subtitle#1{\subsection{#1}}
    \fi
\fi
%%%%%%%%%%%%%%%%%%%%%%%

\maintitle{Covering Groups}
\subtitle{Schur Multiplier}

$A_n$ as quotient group of $2.A_n$.

Let $+\pi$ and $-\pi$ be the two preimages of $\pi\in A_n$ under the natural quotient map. \textcolor{red}{But there is no canonical choice of which element gets which sign.}

{\color{gray}
Let $+1$ be the identity in $2.A_n$. For each $\pi\in A_n$, we define $+\pi$ to be the element which multipilied together with $+1$ gives itself, and $-\pi$ for the other one.
}

\begin{definition}
	$\tilde{G}$ is a \textit{covering group} of $G$ if $Z(\tilde{G})\leq \tilde{G}'$ and $\tilde{G}/Z(\tilde{G})\cong G$. 

    If $|Z(\tilde{G})|=2,3$, then the covering group is called double, triple cover.
\end{definition}

\begin{theorem}[Schur]
	Every finite perfect group $G$ has a unique maximal covering group $\tilde{G}$, with the property that every other covering group is a quotient of $\tilde{G}$. 
    We call $\tilde{G}$ the \textbf{universal covering group} of $G$ and $Z(\tilde{G})$ the \textbf{Schur multiplier} of $G$, denoted as $M(G)$.
\end{theorem}

\begin{example}[non-perfect]
	$\mathbb{Z}_2\times\mathbb{Z}_2=\langle\alpha\rangle\times\langle\beta\rangle$ has four maximal covering groups: one $Q_8$ and three $D_8$. 

    $Q_8=\{\pm 1,\pm i,\pm j,\pm k\}$, $Z(Q_8)=\{\pm 1\}$, $Q_8/Z(Q_8)\cong\mathbb{Z}_2\times\mathbb{Z}_2$.
    
    $D_8=\langle a\rangle:\langle b\rangle$, $Z(D_8)=\langle a^2\rangle$, $D_8/Z(D_8)=\langle Z(D_8)a\rangle\times \langle Z(D_8)ab\rangle \cong\mathbb{Z}_2\times\mathbb{Z}_2$.
    
    Only one of 3 different subgroups of $D_8/Z(D_8)$ could map to the normal subgroup $\langle a\rangle$ of $D_8$. Thus there are 3 different covering groups of $\mathbb{Z}_2\times\mathbb{Z}_2$. However, $Q_8$ has more symmetric structure coinciding with $\mathbb{Z}_2\times\mathbb{Z}_2$. 
\end{example}

\begin{example}
    The alternating groups $A_n$ $(n > 4)$ only have double covers, except for $A_6$ and $A_7$ both with the same Schur multiplier $\mathbb{Z}_6$.
\end{example}

% \href{https://math.stackexchange.com/questions/4242277/number-of-schur-covering-groups}{number of Schur covering groups}



\subtitle{Double Covers of $A_n$ and $S_n$}

Now we define $2.S_n$.

Firstly, let $G$ be a set of order $2n!$, with a map $\varphi$ onto $S_n$ such that each $\pi\in S_n$ has exactly two preimages denoted as $+\pi$ and $-\pi$. 

Intuitively, we should define the multiplication of $G$ as $+\pi+\sigma=+(\pi\sigma)$ and $+\pi-\sigma=-(\pi\sigma)$. 

WLOG, we denote $+(1\ 2)$ as $[1\ 2]$ and $-(1\ 2)$ as $-[1\ 2]$. Then for each transposition $\pi\in S_n$, taking $(+\pi)^{-1}$ to be a preimage of $\pi$, define the products (of 3 elements in $G$) $[i\ j]^{+\pi}$ and $[i\ j]^{-\pi}$ to be a same preimage of $(i^\pi\ j^\pi)$, we denote it as $-[i^\pi\ j^\pi]$. That is $[i\ j]^{\pm\pi}=-[i^\pi\ j^\pi]$. 

Then define the preimage of $(a_i, a_{i+1},\cdots, a_j)$ by \[[a_i\ a_{i+1}\ \cdots\ a_j]=[a_i\ a_{i+1}][a_i\ a_{i+2}]\cdots[a_i\ a_j].\]

Finally by multiplying together disjoint cycles in the usual way, we obtain all elements.

The multiplication defined above is well-defined. That is, if we compute the same product in two different ways, we get the same result. A proof using construction of double cover of orthogonal group is given in Section 3.9.

Note that, by the rule $[i\ j]^{\pm\pi}=-[i^\pi\ j^\pi]$ to define $[i\ j]$'s, all the elements $\pm [i\ j]$ are conjugate. Thus they are all square to 1 or -1 simotaneously. If $(\pm[i\ j])^2=1$, then the elements like $\pm[1\ 2][3\ 4]$ square to -1. And vice versa. 

Therefore, we obtain two distinct double cover of $S_n$. We denote $2.S_n^+$ (actually $\mathbb{Z}_2\times S_n$) as the one with $|[i\ j]|=2$ (where $(i\ j)$ is identified with a coset $Z(2.S_n)a$ with a of order 2), and $2.S_n^-$ as the other one with $|[i\ j]|=4$ ($(i\ j)$ a coset of an element of order 4). However, both of them has the same subgroup $2.A_n\cong \mathbb{Z}_2\times A_n$ of index 2, which is the unique double cover of $A_n$.


\subtitle{Triple Covers of $A_6$ and $A_7$}




%%%%%%%%%%%%%%%%%%%%%%%
\ifx\ChapTwoSecSeven\undefined % Compile separately
     %\bibliography{abcd}
     \end{document}
\fi
