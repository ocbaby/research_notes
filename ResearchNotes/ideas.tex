\documentclass[a4,11pt]{article}

\usepackage[utf8]{inputenc}
\usepackage{amsmath}
\usepackage{amsfonts}
\usepackage{amssymb}
\usepackage{amsthm}
\usepackage{graphicx}
\usepackage{indentfirst}


\newtheorem{theorem}{Theorem}[section]
\newtheorem{corollary}{Corollary}
\newtheorem{proposition}{Proposition}
\newtheorem{lemma}{Lemma}[theorem]
\newtheorem{conjecture}{Conjecture}[section]
\newtheorem{definition}{Definition}
\newtheorem{problem}{Problem}[section]
\newtheorem{example}{Example}
\newtheorem{remark}{Remark}


\def\PSL{\mathrm{PSL}}
\def\PGL{\mathrm{PGL}}
\def\PSU{\mathrm{PSU}}
\def\PSp{\mathrm{PSp}}
\def\PO{\mathrm{P\Omega}}
\def\PSL{\mathrm{PSL}}
\def\Aut{\mathrm{Aut}}
\def\Inn{\mathrm{Inn}}
\def\Out{\mathrm{Out}}
\def\Cay{\mathrm{Cay}}



\title{Research Ideas on Symmetrical Graph Theory}
\author{Yuandong Li}
\date{\today}


\begin{document}

\maketitle

\tableofcontents
\newpage


%%%%%%%%%%%%%%%%%%%%%%
\section{construction of 2-arc-transitive non-Cayley normal cover of $K_{2^n,2^n}$}
paper3


%%%%%%%%%%%%%%%%%%%%%%
\section{characterize locally 2-arc-transitive graphs of order $p^n$}
paper-two



%%%%%%%%%%%%%%%%%%%%%%
\section{$\Cay(G,S)$ with $\Aut(G,S)$ 2-transitive on $S$}

\begin{theorem}[Godsil,1981]
	$N_{\Aut\Gamma}(\hat{G})=\hat{G}\rtimes \Aut(G,S)$
\end{theorem}

\begin{conjecture}[Xu, 1998]
	Almost all Cayley graphs are normal.
\end{conjecture}


\begin{lemma}
	If $\Aut(G,S)$ is s-transitive on $S$, then $\Gamma$ is $(N_{\Aut\Gamma}(\hat{G}),s)$-arc-transitive.
\end{lemma}

\begin{theorem}[Praeger,1999]
	cubic normalizer-arc-transitive $\Cay(G,S)$ on nonab. simple $G=T$ satisfies $\Aut\Gamma=\hat{G}\rtimes \Aut(G,S)$
\end{theorem}

\begin{theorem}[Li,1998]
	normalizer-2-arc-transitive  $\Cay(G,S)$ on nonab. char. simple $G=T^l$ satisfies $\Aut\Gamma=\hat{G}\rtimes \Aut(G,S)$
\end{theorem}

\begin{conjecture}
	$\Gamma=\Cay(G,S)$ is connected. If $\Aut(G,S)$ is 2-transitive on $S$, then either $\Aut\Gamma=\hat{G}\rtimes \Aut(G,S)$, or $\Gamma$ is a normal cover of $\mathbf{K}_{p^e,p^e}$
\end{conjecture}

For solvable $G$, Zhou find exceptional normal cover of $\mathbf{K}_{2^f}$ with $f\geq 3$.

\begin{problem}
	nonsolvable $G$
\end{problem}

\begin{problem}
	2-transitive $\rightarrow$ (quasi)primitive
\end{problem}



%%%%%%%%%%%%%%%%%%%%%%
\section{bi-quasiprimitive 2-arc-transitive graph}
see Praeger, Ivanov 1993: affine (bi-)primitive 2-arc-transitive graphs
\begin{problem}
	Generalize to (bi-)quasiprimitive 2-arc-trans. graphs
\end{problem}


%%%%%%%%%%%%%%%%%%%%%%
\section{locally primitive (s-arc-transitive) Cayley graphs}
[cyclic, dihedral, solvable, alternating, simple] 

more complex $G$? bi-Cayley?

\subsection{arc-transitive (bi-)Cayley graphs}
arc-transitive Cayley graphs on T, odd prime val with solvable stabilizers

Extended to bi-Cayley with same condition, while focus on bipartite

Extended to bi-Cayley of bi-quasiprimitive instead of solvable stabilizer


\begin{problem}[*]
	Study arc-transitive bi-quasiprimitive bi-Cayley graphs on T of val=pq or $p^e$
\end{problem}

release Lemma 2.4 ($G_v$ solvable $\rightarrow$ $G_{vu}^{[1]}=1$)

such that Lemma 3.3 can be extended using Li-Wang-Xia


\subsection{s-arc-transitive solvable Cayley graphs}
$s\leq 2$ sharply for non-bipartite
$s\leq 4$ sharply for bipartite
\begin{problem}[*]
	Study locally (quasi)primitive solvable Cayley graphs.
\end{problem}

locally primitive graphs of order $p^e$ are studied by Li,Pan,Ma 2009

locally primitive metacyclic normal Cayley graphs are studied by Pan 09

locally primitive abelian Cayley graphs are studied by Li,Lou,Pan 2011

locally primitive dihedral Cayley graphs are studied by Pan, 2014


\subsection{s-arc-transitive Cayley graphs on non-solvable groups}
Locally primitive (s-arc-transitive) Cayley graphs of finite simple groups is classified by Li,Wang,Xia

\textbf{method:}

$\bar{X}:=X/M$ quasiprimitive, reduct to AS

$\bar{X}=\bar{G} \bar{X}_\alpha$, use factorization of AS groups

If $\bar{G}$ or $\bar{X}_\alpha$ is solvable, then use Li-Xia. 

Else both non-solvable, use Li-Wang-Xia. 

\begin{problem}(*)
	Locally primitive (s-arc-transitive) Cayley graphs on $T^l$
\end{problem}

%%%%%%%%%%%%%%%%%%%%%%
\section{Weiss's Conjecture}

\begin{conjecture}
	Suppose $\Gamma = (V, E)$ is a finite undirected graph, $G\leq \Aut\Gamma$ such that G
is vertex-transitive and locally primitive. Then $|G_x|$ is bounded by a number depending only on $|\Gamma(x)|$.
\end{conjecture}

Now, in addition, $\Gamma$ is (G,s)-transitive. Then the number of locally s-arcs divides $|G_x|$. But this bound on s is dependent on $|\Gamma(x)|$.

\begin{problem}[open?]
	bound $s\leq 3$ for $|\Gamma(x)|\geq 3$ and $G_{x,y}^{[1]}=1$.
\end{problem}


structure of $G_x/O_p(G_x^{[1]})$ described for arc-trans loc. qp, especially $G_x$ for arc-trans. loc. 2-trans.

\begin{problem}
	Suppose $\Gamma = (V, E)$ is a finite undirected graph, $G$ is transitive on V and s-arcs respectively with $G_{x,y}^{[1]}=1$.\\Determine the upper bound on s.
\end{problem}

s-arc-transitive graphs with $s\geq 2$ are locally-primitive.


%%%%%%%%%%%%%%%%%%%%%%
\section{arc-transitive locally quasiprimitive graph}

Graphs mentioned here are locally finite G-graphs, i.e., vertex stabilizers $G_\alpha$ are finite.

\subsection{locally quasiprimitive G-graphs}
\begin{problem}
	Determine the structure of $G_x$.
\end{problem}

$\because$ locally quasiprimitivity $\implies$ edge-transitivity, 

$\therefore G\curvearrowright V\Delta$ has at most 2 orbits (adjacent vertices in distinct orbits).

\begin{problem}
	Bound $|G_\alpha|$ and determine amalgams $(G_\alpha,G_\beta;G_{\alpha,\beta})$
\end{problem}

\begin{itemize}
	\item Tutte: bounded $|G_x|$ when G is vertex-transitive on finite trivalent graphs
	\item Goldschmidt: bounded $|G_x|$ for locally finite trivalent G-graphs, determined amalgams when locally primitive
	\item locally finite and locally s-arc transitive graphs with $s\geq 6$ in which each vertex has val$\geq 3$ the amalgams are weak BN-pairs, known.
	\item for $s\geq 4$ a partial classification of the amalgams was recently obtained 
\end{itemize}


general results on the structure of $G_x$: Thompson-Wielandt Theorem.

\centerline{
G loc. fin. + loc. q-prim. + V-trans. $\mathop{\Longrightarrow}\limits^{T-W}$ $G_{x,y}^{[1]}$ is a p-group}

\subsection{arc-trans. loc. q-prim. G-graphs}

Now assume $\Delta$ loc. fin. + loc. q-prim. + arc-trans.


In [19] Weiss conjectured that the order of $O_p(G_x^{[1]})$ is bounded by a function of $|\Delta(x)|$ when $G_x^\Delta$ is also a primitive permutation group on $\Delta$.

\begin{problem}
	Obtain the structure of $O_p(G_x^{[1]})$ or bound its order, assuming something on $G_x^{\Delta(x)}$ and its action.
\end{problem}

Amalgams of arc-trans. loc. 2-trans. graphs with $G_{x,y}^{[1]}\neq 1$ studied.

\begin{problem}
	Determining amalgams of arc-transitive G-graphs of a given valency. (recently)
\end{problem}

raises
\begin{problem}
	Study $G_x^{[1]}$ when $G_{x,y}^{[1]}= 1$, and about $G_x^{[1]}/O_p(G_x^{[1]})$ in general.
\end{problem}

This paper gives
\begin{theorem}
	...
\end{theorem}
\begin{theorem}
	describes the structure of $G_x/O_p(G_x^{[1]})$
\end{theorem}
applied to obtain 
\begin{theorem}
	the structure of $G_x$ of a class of arc-trans. loc. fin. and loc. 2-trans. G-graphs with trivial edge kernel.
\end{theorem}


%%%%%%%%%%%%%%%%%%%%%%
\section{Homogenous graphs}
works on this area usually published on top journals such as JCTB
\subsection{construction of TW type 3-connected homogeneous graph}
3-ch



%%%%%%%%%%%%%%%%%%%%%%
\section{Generalized quadrangles}




\end{document}
