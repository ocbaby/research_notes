\documentclass[a4paper,11pt]{article}



\usepackage[utf8]{inputenc}
\usepackage{amsmath}
\usepackage{amsfonts}
\usepackage{amssymb}
\usepackage{amsthm}
\usepackage{graphicx}
\usepackage{enumerate}
\usepackage{geometry}
\usepackage{hyperref}
\usepackage{xcolor}
\usepackage{indentfirst}
\usepackage{enumitem}


\geometry{
  paper=a4paper,
  margin=64pt,
  %includeheadfoot
}
\linespread{1.3}
\setlength{\parskip}{3pt}





\newtheorem{theorem}{Theorem}[section]
\newtheorem{corollary}[theorem]{Corollary}
\newtheorem{proposition}[theorem]{Proposition}
\newtheorem{lemma}[theorem]{Lemma}
\newtheorem{conjecture}[theorem]{Conjecture}
\newtheorem{definition}[theorem]{Definition}
\newtheorem{example}[theorem]{Example}
\newtheorem{remark}[theorem]{Remark}




\def\SL{\mathrm{SL}}
\def\GL{\mathrm{GL}}
\def\PSL{\mathrm{PSL}}
\def\PGL{\mathrm{PGL}}
\def\PSU{\mathrm{PSU}}
\def\PSp{\mathrm{PSp}}
\def\PO{\mathrm{P\Omega}}
\def\Aut{\mathrm{Aut}}
\def\Inn{\mathrm{Inn}}
\def\Out{\mathrm{Out}}
\def\Cay{\mathrm{Cay}}


\usepackage{indentfirst}

\title{Complement Product of nonabelian simple group and cyclic group}
\author{Yuandong Li}
\date{\today}


\begin{document}

%\maketitle

%\tableofcontents

\section{Introduction}

I got this topic from a report entitled ``Complementary product of cyclic group and dihedral group" given by Prof. Young Soo Kwon at BJTU. Here is the abstract of the report:
\begin{quote}
	A finite group G is a complementary product of two subgroups A and B if $G=AB$ and $A\cap B = {1_G}$. In this talk, for all odd numbers $k\geq 3$, using the theory of skew morphisms and associated extended power functions, we present a classification of complementary products of a cyclic group and a dihedral group $D_{2k}$. As an application, the regular generalized Cayley maps of odd valency over cyclic groups are classified. We also suggest some open problems related to complementary product of cyclic group and dihedral group. This is a joint work with Istvan Kovacs and Kan Hu.
\end{quote}

In the end, Prof. Kwon listed several problems such as 
\begin{quote}
	Classify complementary product group $\Gamma=GY$, where \\G: nonabelian finite simple group, Y: cyclic group.
\end{quote}



\section{Preliminaries}

Let $\Gamma=GY$ where $Y=\langle y\rangle\cong\mathbb{Z}_n$ and $G\cap Y=1$.

Then $\Gamma=\{gy^i\mid g\in G, 1\leq i\leq n\}=\{y^ig\mid g\in G, 1\leq i\leq n\}$ 

and $gy^i=hy^j\iff g=h$ and $i=j$ (since $G\cap Y=1$).

For any $g\in G$, $yg=hy^i$ for some $h\in G$ and $1\leq i\leq n$.

Then $\phi(g):=h$ and $\Pi(g):=i$ are well-defined. 

Then $\phi: G\to G$, $\Pi:G\to [n]$\textcolor{red}{(could gen. to Y?)} and 
\[\phi(gh)y^{\Pi(gh)}=ygh=\phi(g)y^{\Pi(g)}h=\phi(g)y^{\Pi(g)-1}\phi(h)y^{\Pi(h)}=...=\phi(g)\phi^{\Pi(g)}(h)y^{\Pi(h)+\Pi(\phi(h))+...+\Pi(\phi^{\Pi(g)-1}(h))}\]
\[ \phi(gh)=\phi(g)\phi^{\Pi(g)}(h)\text{ and }\Pi(gh)=\Pi(h)+\Pi(\phi(h))+...+\Pi(\phi^{\Pi(g)-1}(h))=:\sigma_\Pi(h,\Pi(g)). \]

Let $m=|\phi|$ and $\pi(g):\equiv\Pi(g)\mod m$. Then $\pi(g)\in\{0,1,2,...,m-1\}$, $\phi(gh)=\phi(g)\phi^{\pi(g)}(h)$.

\noindent\textbf{Remark:}
\begin{enumerate}
	\item $m|n$ : {$gy^n=g=y^ng=y^{n-1}\phi(g)y^{\Pi(g)}=...=\phi^n(g)y^{\Pi(g)+\Pi(\phi(g))+...+\Pi(\phi^{n-1}(g))}\Rightarrow g=\phi^n(g)$} \\
		  $\mathrm{core}_\Gamma(Y)=\langle y^m\rangle$ :
		{$y^{kx}=g^{-1}y^xg=g^{-1}\phi^{x}(g)y^{\Pi(g)+\Pi(\phi(g))+...+\Pi(\phi^{x-1}(g))}\Rightarrow x\equiv 0\mod m$}
\end{enumerate}

\end{document}



