\documentclass[a4,11pt]{article}

\usepackage[utf8]{inputenc}
\usepackage{amsmath}
\usepackage{amsfonts}
\usepackage{amssymb}
\usepackage{amsthm}
\usepackage{graphicx}
\usepackage{enumerate}


\newtheorem{theorem}{Theorem}[theorem]
\newtheorem{corollary}{Corollary}[theorem]
\newtheorem{proposition}{Proposition}[theorem]
\newtheorem{lemma}{Lemma}[theorem]
\newtheorem{conjecture}{Conjecture}[theorem]
\newtheorem{definition}{Definition}[theorem]
\newtheorem{example}{Example}[theorem]
\newtheorem{remark}{Remark}[theorem]


\def\PSL{\mathrm{PSL}}
\def\PGL{\mathrm{PGL}}
\def\PSU{\mathrm{PSU}}
\def\PSp{\mathrm{PSp}}
\def\PO{\mathrm{P\Omega}}
\def\PSL{\mathrm{PSL}}
\def\Aut{\mathrm{Aut}}
\def\Inn{\mathrm{Inn}}
\def\Out{\mathrm{Out}}
\def\Cay{\mathrm{Cay}}





\title{Notes on GTM251}
\author{Yuandong Li}
\date{\today}


\begin{document}

\maketitle

\tableofcontents
\newpage

\section{Introduction}
\subsection{History}
\subsection{CFSG}
Every finite simple group is isomorphic to one of the followings:
\begin{enumerate}[(i)]
	\item a cyclic group $C_p$ of prime order $p$;
	\item an alternating group $A_n$ for $n\geq 5$;
	\item a classical group:
		\begin{itemize}
			\item linear: $\PSL_n(q)$, $n\geq 2$, except $\PSL_2(2)$ and $\PSL_2(3)$;
			\item unitary: $\PSU_n(q)$, $n\geq 3$, except $\PSU_3(2)$;
			\item symplectic: $\PSp_{2n}(q)$, $n\geq 2$, except $\PSp_4(2)$;
			\item orthogonal: $\PO_{2n+1}(q)$, $n\geq 3$, $q$ odd; $\PO_{2n}^+(q)$, $\PO_{2n}^-(q)$, $n\geq 4$;
		\end{itemize}
		where $q$ is a power $p^a$ of a prime $p$;
	\item an exceptional group of Lie type:
		\[ G_2(q),q\geq 3;\ F_4(q);\ E_6(q);\  ^2E_6(q);\ ^3D_4(q);\ E_7(q);\ E_8(q) \] with $q$ a prime power, or\[ ^2B_2(2^{2n+1}),\  ^2G_2(3^{2n+1},\ ^2F_4(2^{2n+1}),\ n\geq 1; \]or the Tits group $^2F_4(2)'$;
	\item one of 26 sporadic simple groups:
		\begin{itemize}
			\item the five Mathieu groups $M_{11}$, $M_{12}$, $M_{22}$, $M_{23}$, $M_{24}$;
			\item the seven Leech lattice groups Co$_1$, Co$_2$, Co$_3$, McL, HS, Suz, J$_2$;
			\item the three Fischer groups Fi$_{22}$ , Fi$_{23}$, Fi$_{24}'$ ;
			\item the five Monstrous groups $\mathbb{M}$, $\mathbb{B}$, Th, HN, He;
			\item the six pariahs J$_1$, J$_3$, J$_4$, O'N, Ly, Ru.
		\end{itemize}
\end{enumerate}
Conversely, every group in this list is simple, and the only repetitions in this list are:
\begin{equation*}
\begin{aligned}
	\PSL_2(4)\cong &\PSL_2(5)\cong A_5;\\
	&\PSL_2(7)\cong\PSL_3(2);\\
	&\PSL_2(9)\cong A_6;\\
	&\PSL_4(2)\cong A_8;\\
	&\PSU_4(2)\cong \PSp_4(3).
\end{aligned}
\end{equation*}

introduce, construction, orders, simplicity, action(reveal subgroup structure)

\subsection{App of CFSG}
The symmetric difference set of almost simple subgroups and maximal subgroups of $A_n$ is listed out, while listing their intersection is impossible.

\section{The Alternating Groups}
\subsection{Introduction}
$\Aut(A_n)\cong S_n$ for $n\geq 7$ but for $n=6$ there is an exceptional outer automorphism of $S_6$.

subgroup structure (O'Nan-Scott Thm)



\end{document}





